Un Wiki es un sitio web que puede ser modificado por múltiples personas. Generalmente los wikis son de carácter informativo, un claro ejemplo es Wikipedia, que es la enciclopedia online más popular del mundo basada en el concepto wiki. Al ser un sitio colaborativo es necesario llevar un historial de los cambios realizados por los usuarios, logrando un control en las ediciones de estos. De una edición se pueden sacar algunas propiedades importantes como: el autor de la edición, que se representa con una dirección IP cuando es anónimo y con un nombre de usuario en caso contrario, la cantidad de texto editado y la fecha y hora en que realizó la edición. Cabe destacar, que dicho historial de ediciones es de suma importancia para aquellas personas que le hacen seguimiento o que de alguna forma les interesa el estado del artículo, estas personan son consideradas \textit{watchers} del artículo.

Hoy en día existen cantidades de personas colaborando en estos sitios que hacen que el historial de ediciones se haga suficientemente extenso y difícil de comprender. La abundancia de datos provoca complejidad en su búsqueda e interpretación, lo que da lugar a la necesidad de un mecanismo que permita facilitar la transmisión y comprensión de la información, llamado visualización de datos.

La visualización de datos logra transmitir un conjunto inmenso de datos de manera clara y lo hace como su nombre indica, a través de elementos visuales, es decir, gráficas que combinan variedad de colores, figuras y texto. Es importante destacar que la visualización de datos necesita un estudio previo para la preparación, transformación y análisis de los datos. Debido a lo expuesto anteriormente, en este trabajo se propone la implementación de una herramienta web que consta de un editor de visualizaciones de propiedades de historiales de wikis, en donde los datos necesarios para las visualizaciones serán surtidos principalmente por un servicio (API) llamado Wikimetrics 2.0. 

\section{Objetivo general}
Desarrollar una aplicación web que permita construir y editar visualizaciones de propiedades de historiales de wikis.

\section{Objetivos específicos}
\begin{itemize}
\item{Diseñar visualizaciones generales basadas en la información de los historiales de artículos de wikis provista por el API de Wikimetrics 2.0.}
\item{Definir los requerimientos de la aplicación.}
\item{Implementar una interfaz SPA adaptativa que ofrezca las funcionalidades requeridas por un \textit{watcher} de un wiki.}
\item{Implementar un servicio API que delegue los requerimientos de la aplicación en cuanto a persistencia de datos.}
\item{Utilizar un método ágil para el desarrollo de la aplicación.}
\item{Realizar el despliegue y puesta en producción de la aplicación.}
\end{itemize}

\section{Justificación}
La justificación de este trabajo recae en la posibilidad de hacer investigación en un campo que está siendo cada vez más explorado que es la visualización de datos, que desencadena el área de analistas de datos, y por otro lado un área sumamente amplia que es el desarrollo en tecnologías de internet. 

Este trabajo va dirigido especialmente para aquellas personas que le hacen seguimiento a artículos de wikis y quieren informarse rápidamente de anomalías, cambios e información de interés sobre dichos artículos, con el resultado de este trabajo se facilitará mucho más su trabajo, logrando así un artículo de mayor calidad.

Adicionalmente este trabajo puede servir como base para futuros Trabajos Especiales de Grados en la Escuela de Computación de la Facultad de Ciencias en la Universidad Central de Venezuela relacionados con visualización de datos y tecnologías en el área web.

\section{Distribución del documento}
El presente trabajo se encuentra dividido en cinco (5) capítulos. En donde, el capítulo 1, introduce el contexto, el problema, los objetivos planteados (general y específicos), la justificación de la investigación y la distribución del documento. El capítulo 2, presenta las bases teóricas sobre Wiki y su entorno, y la visualización de datos, en donde dichos conceptos son necesarios para lograr el entendimiento de capítulos posteriores. El capítulo 3 presenta la investigación y evaluación de herramientas de apoyo para el desarrollo del proyecto. El capítulo 4 constituye el análisis e interpretación  de los resultados presentados en las actividades aplicadas para alcanzar los objetivos planteados. Por último, el Capítulo 5 presenta las conclusiones del trabajo realizado, describiendo los aportes logrados, limitaciones encontradas y planteamiento de trabajos futuros.