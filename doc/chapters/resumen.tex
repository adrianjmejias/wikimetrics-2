\textbf{Título:}\\
Desarrollo de un editor de visualizaciones de propiedades de historiales de wikis.

\textbf{Autor:}\\
Leonardo Testa.

\textbf{Tutor:}\\
Prof. Eugenio Scalise.

Un Wiki es un sitio web, generalmente de carácter informativo (como lo es Wikipedia), que puede ser modificado por múltiples personas. Cada una de estas modificaciones son almacenadas, y en conjunto conforman un historial de versiones, en donde cada versión representa una modificación y los efectos que causó en el artículo wiki.
Siendo Wikipedia un caso real con bastante popularidad, es normal que el historial de versiones de un artículo sea suficientemente extenso y complejo, por lo tanto las personas interesadas en mantener el artículo \textit{“sano”} perderán una gran suma de tiempo revisando las modificaciones.
En este documento, presentaremos la investigación y la realización de una herramienta web que facilita la lectura de propiedades del historial a aquellas personas interesadas, en donde se optará por visualización de datos como estrategia, de esta forma, mediante una interfaz capaz de manipular gráficas el usuario podrá proyectar distintas propiedades y conseguir fácilmente información más completa y concretar patrones.


\textbf{Palabras claves:}\\
Visualización de datos, wiki, propiedades de historiales, gráficas, herramienta web, editor de visualizaciones, wikipedia.
